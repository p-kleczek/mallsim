\chapter{Wprowadzenie}
\label{cha:wprowadzenie} 
 
 Zagadnienie modelowania ruchu i zachowania pieszych jest od d�ugiego czasu postrzegane jako istotne z punktu widzenia sektora us�ugowego i handlowego. Przekonanie to znajduje potwierdzenie w badaniach przytoczonych cho�by  przez Aloysa Borgersa i Harry'ego Timmermansa \cite{BoTi86-1}, kt�re wykaza�y wysok� zale�no�� rentowno�ci sklep�w od sposobu poruszania si� potencjalnych klient�w. Problem ten jest do�� z�o�ony i dodatkowo mo�e by� rozwa�any na r�nych poziomach abstrakcji, w zale�no�ci od przeznaczenia modelu.
 
%---------------------------------------------------------------------------

\section{Cele pracy}
\label{sec:celePracy}

Celem niniejszego projektu jest opracowanie modelu ruchu ludzi w centrum handlowym oraz stworzenie na jego podstawie symulacji komputerowej, kt�ra pozwoli oceni� jako�� modelu.

%---------------------------------------------------------------------------

\section{Zawarto�� pracy}
\label{sec:zawartoscPracy}

G��wna cz�� pracy sk�ada si� z pi�ciu rozdzia��w. W rozdziale 2 dokonano przegl�du r�nych podej�� do zagadnienia modelowania przemieszczania si� pieszych. W rozdziale 3 przedstawiono stworzony na potrzeby projektu model centrum handlowego. W rozdziale 4 om�wiono algorytmy stosowane do modelowania ruchu pieszych. Rozdzia� 5 zawiera opisy szczeg��w implementacyjnych. Rozdzia� 6 prezentuje wyniki symulacji i ich analiz�.
